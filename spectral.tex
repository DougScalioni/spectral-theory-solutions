\documentclass{article}
\usepackage[utf8]{inputenc}
\usepackage{amssymb}
\usepackage{amsmath}
\usepackage{amsthm}
\usepackage{xcolor}


\newtheoremstyle{exercisestyle}
  {12pt} % Space above
  {\topsep} % Space below
  {} % Body font
  {} % Indent amount
  {\itshape} % Theorem head font
  {.} % Punctuation after theorem head
  {.5em} % Space after theorem head
  {} % Theorem head spec (can be left empty, meaning `normal')

\theoremstyle{exercisestyle}
\newtheorem{innerex}{Exercise}
\newenvironment{exercise}[1]
  {\renewcommand\theinnerex{#1}\innerex}
  {\endinnerex}

\DeclareMathOperator{\domain}{dom } \newcommand{\dom}{\domain}

\DeclareMathOperator{\range}{rng } \newcommand{\rng}{\range}

\DeclareMathOperator{\nullspace}{N} \newcommand{\kernel}{\nullspace}

\DeclareMathOperator{\bounded}{B}

\newcommand{\norm}[1]{\lVert #1 \rVert}
\newcommand{\closure}[1]{\overline{ #1}}
\newcommand{\inner}[2]{\langle #1, #2 \rangle}
\newcommand{\openball}[2]{B\left(#1;#2\right)}
\newcommand{\ball}[2]{\openball{#1}{#2}}
\newcommand{\closedball}[2]{B\left[#1;#2\right]}




\DeclareMathOperator{\Natural}{\mathbb{N}}
\DeclareMathOperator{\Integer}{\mathbb{Z}}
\DeclareMathOperator{\Rational}{\mathbb{Q}}
\DeclareMathOperator{\Real}{\mathbb{R}}
\DeclareMathOperator{\Complex}{\mathbb{C}}
\DeclareMathOperator{\field}{\mathbb{F}}
\DeclareMathOperator{\banach}{\mathcal{B}}
\DeclareMathOperator{\hilbert}{\mathcal{H}}
\DeclareMathOperator{\normed}{\mathcal{N}}
\DeclareMathOperator{\graph}{\mathcal{G}}
\DeclareMathOperator{\id}{\textbf{1}}
\DeclareMathOperator{\strongly}{\stackrel{s}{\longrightarrow}}
\DeclareMathOperator{\weakly}{\stackrel{w}{\longrightarrow}}
\DeclareMathOperator{\linspan}{\text{Lin}}



\renewcommand{\labelenumi}{\alph{enumi})}


\begin{document}


\title{Intermediate Spectral Theory and Quantum Dynamics \\ César R. de Oliveira \\ Notes and Solutions}
\author{Douglas Scalioni Domingues}
\date{April 2021}

\maketitle

\section{Linear Operators and Spectra}

\subsection{Bounded Operators}

\begin{exercise}{1.1.5}
    Let $T : \dom{T} \subset X \rightarrow Y$ be a linear operator. Verify the following
    items:

    \begin{enumerate}
        \item The range of T , $\rng T := T (\dom T) \subset Y$, and the kernel (or null space) of T,
              $\kernel(T) := \{\xi \in \dom T : T \xi = 0\}$, are vector spaces.


              \underline{rng}: Let $X$, $Y$ be in the range of $T$ and $\alpha \in \mathbb{F}$. There
              exist $x$ and $y$ in $\dom T$ such that $Tx=X$ and $Ty = Y$. Since $\dom T$ is a vector
              subspace, $\alpha x + y$ is also in $\dom T$, and therefore $T(\alpha x + y)$ exists. $T$
              is linear, then $T(\alpha x + y) = \alpha T(x) + T(y)$.


              \underline{kernel}:
              Let $x, y \in \textrm{ker}(T)$ and $\alpha \in \mathbb{F}$. Then $T(\alpha x+y) = \alpha T(x) + T(y) = \alpha 0 + 0 = 0$,
              and $\alpha x + y \in \textrm{ker}(T)$.


        \item If the dimension $\dim(\dom T) = n < \infty$, then $\dim(\rng T ) \leq n$.


              The dimension of a vector space is given by the number of elements in its basis. Suppose $B = \{ b_1, ..., b_n\}$
              is a basis of dom $T$ with $n$ elements. Given an arbitrary element x of $\dom T$, we have $x = a_1 b_1 + ... + a_n b_n$.
              The range of $T$ is completely determined by its action on the basis of $\dom T$:
              \begin{math}
                  T(x) = T(a_1 b_1 + ... + a_n b_n) = a_1 T(b_1) + ... + a_n T(b_n)
              \end{math}
              Clearly, the set $\{T(b_1), ... , T(b_n)\}$ has $n$ elements and spans rng $T$, and therefore  $\dim(\rng T) \leq n$.


        \item The inverse operator of $T$, $T^{-1}: \rng T$ $\rightarrow \dom T$, exists if, and only if, $T\xi = 0 \Rightarrow \xi = 0$
              and, in case it exists, it is also a linear operator.


              For $T$ to have an inverse it needs to be a bijection. In particular, $T$ has to be injective (which means that $T(x) = T(y)$ implies that $x=y$ $\forall x, y$).
              Let $a,b \in \dom T$ such that $T(a) = T(b) \Rightarrow T(a) - T(b) = T(a-b) = 0$. Clearly, $T$
              is injective \textit{iff} $a-b = 0$.

              $T$ is guaranteed to be surjective since it's defined onto its range. Calling $a-b = \xi$, we have
              $$ T \text{ is invertible} \Leftrightarrow T \text{ is bijective} \Leftrightarrow (T\xi = 0 \Rightarrow \xi = 0). $$

        \item  If $T, S$ are invertible linear operators, then $(T S)^{-1} = S^{-1} T^{-1}$ (by supposing,
              of course, that the operations are well posed).

              Of course $(T S)^{-1} TS = I$, so multiplying both sides (on the right) by $S^{-1}$ we get:
              $$ (T S)^{-1} T(S S^{-1}) = S^{-1} $$
              $$ (T S)^{-1} T = S^{-1} $$
              Similarly with $T^{-1}$,
              $$ (T S)^{-1}(T T^{-1}) = S^{-1}T^{-1}$$
              $$ (T S)^{-1} = S^{-1}T^{-1}.$$

    \end{enumerate}

\end{exercise}

\begin{exercise}{1.1.9}
    Let $X$ and $Y$ be finite-dimensional vector spaces and $T : X \to Y$
    a linear operator. Choose bases in $X$ and $Y$ and show that $T$ can be represented
    by a matrix, and discuss how the matrix that represents $T$ changes if other bases
    are considered.

    It's given that $X$ and $Y$ are finite-dimensional, so let $B_x = \{ x_1, ..., x_n \}$ and $B_y = \{y_1, ..., y_m\}$ be their basis, respectively.
    Consider now an arbitrary element $\xi \in X$, which has an unique representation in terms of $B_x$: $\xi = \alpha_1 x_1 + ... + \alpha_n x_n$.
    Applying the transformation $T$, we have:
    $$ T(\xi) = T(\alpha_1 x_1 + ... + \alpha_n x_n) = \alpha_1 T(x_1) + ... + \alpha_n T(x_n)$$
    We can see, then, that the transformation $T$ is determined by its action on the basis elements of $X$. Each $T(x_i)$ has an unique representation in terms of $B_y$:
    $T(x_i) = \beta_{1i} y_1 + ... + \beta_{mi} y_m$, for $i = 1, ..., n$.
    Considering $T$, again:
    $$ T(\xi) = \sum_{i=1}^n \alpha_i T(x_i) = \sum_{i=1}^n \alpha_i  \sum_{j=1}^m  \beta_{ji} y_j $$
    $$ = \sum_{j=1}^m ( \sum_{i=1}^n  \beta_{ji} \alpha_i ) y_j $$
    \[ = \begin{pmatrix}
            \beta_{11} & \dots  & \beta_{1n} \\
            \vdots     & \ddots &            \\
            \beta_{m1} &        & \beta_{mn}
        \end{pmatrix}
        \begin{pmatrix}
            \alpha_{1} \\
            \vdots     \\
            \alpha_{n} \\
        \end{pmatrix}
        \begin{pmatrix}
            y_1 & \dots & y_m
        \end{pmatrix}
    \]

    Or, in terms of coordinates ($\gamma_i$ being the coordinates of $T(\xi)$ in $Y$):

    \[ \begin{pmatrix}
            \gamma_{1} \\
            \vdots     \\
            \gamma_{n} \\
        \end{pmatrix}_{B_y}
        = \begin{pmatrix}
            \beta_{11} & \dots  & \beta_{1n} \\
            \vdots     & \ddots &            \\
            \beta_{m1} &        & \beta_{mn}
        \end{pmatrix}
        \begin{pmatrix}
            \alpha_{1} \\
            \vdots     \\
            \alpha_{n} \\
        \end{pmatrix}_{B_x}
    \]

    and
    \[ T = \begin{pmatrix}
            \beta_{11} & \dots  & \beta_{1n} \\
            \vdots     & \ddots &            \\
            \beta_{m1} &        & \beta_{mn}
        \end{pmatrix}.
    \]

\end{exercise}


\begin{exercise}{1.1.13} Bounded operators.

    \begin{enumerate}
        \item If $T\in \bounded (\mathcal{N}_1, \mathcal{N}_2)$, check that

              $$ \lVert T \rVert \stackrel{(1)}{=} \inf_{C>0} \{ \lVert T\xi \rVert \leq C \lVert \xi \rVert , \forall \xi \in \mathcal{N}_1 \}
                  \stackrel{(2)}{=} \sup_{\lVert \xi \rVert = 1} \lVert T \xi \rVert
                  \stackrel{(3)}{=} \sup_{\xi \neq 0} \frac{\lVert T \xi \rVert}{\lVert \xi \rVert}
              $$

              It's easiest to start with (2). By definition, $\lVert T \rVert = \sup_{\lVert \xi \rVert \leq 1} \lVert T \xi \rVert$. Let $\xi$ such that $\lVert \xi \rVert < 1$.
              Then, clearly, $$\lVert T \xi \rVert < \lVert T\left( \frac{\xi}{\lVert \xi \rVert} \right) \rVert = \frac{1}{\lVert \xi \rVert}  \lVert T \xi \rVert.$$

              Thus $\sup_{\lVert \xi \rVert < 1} \lVert T \xi \rVert \leq \sup_{\lVert \xi \rVert = 1} \lVert T \xi \rVert$.
              This means that we only need to look at the $\xi \in \mathcal{N}_1$ such that $\lVert \xi \rVert = 1$,
              and therefore $\lVert T \rVert = \sup_{\lVert \xi \rVert = 1} \lVert T \xi \rVert$ is a perfectly equivalent definition of the norm of
              $\lVert T \rVert$, and (2) holds.

              From (2), we just need to note that (obviously) $\frac{\xi}{\lVert \xi \rVert}$ has norm 1. Then,
              $$ \sup_{\lVert \xi \rVert = 1} \lVert T \xi \rVert = \sup_{\xi \neq 0} \lVert T \left(\frac{\xi}{\lVert \xi \rVert}\right) \rVert
                  = \sup_{\xi \neq 0}  \frac{\lVert T \xi \rVert}{\lVert \xi \rVert} , $$
              obtaining (3).

              To obtain (1), recall that $\lVert T \rVert = \sup_{\xi \in \mathcal{N}_1} \lVert T \left( {\xi}/{\lVert \xi \rVert} \right) \rVert$. This means that given $\epsilon > 0$, there exists $\xi$
              such that $\lVert T \left( {\xi}/{\lVert \xi \rVert} \right) \rVert > \lVert T \rVert - \epsilon $, or $$\lVert T \left( {\xi}/{\lVert \xi \rVert} \right) \rVert + \epsilon > \lVert T \rVert.$$
              At the same time, $$\lVert T \left( {\xi}/{\lVert \xi \rVert} \right) \rVert \leq \lVert T \rVert.$$
              Thus, we conclude that
              $$\lVert T \rVert = \inf_{C>0} \{ \lVert T \left( {\xi}/{\lVert \xi \rVert} \right) \rVert \leq C\}
                  = \inf_{C>0} \{ \lVert T \xi \rVert \leq C {\lVert \xi \rVert} , \forall \xi \in \mathcal{N}_1 \}.$$

        \item If $T, S$ are bounded linear operators and $T S$ (the composition, but usually called
              product of operators) is defined, show that $T S$ is bounded and $\lVert T S \rVert \leq \lVert T \rVert \lVert S \rVert $ .
              Therefore, if $T^n$ (nth iterate of $T$) is defined, then $\lVert T ^n \rVert \leq \lVert T \rVert ^n$ .

              Let $\xi \in \dom S$, so $ \lVert T(S(\xi)) \rVert \leq  \lVert T \rVert  \lVert S(\xi) \rVert \leq \lVert T \rVert  \lVert S \rVert \lVert \xi \rVert $.
              Dividing both sides by $\lVert \xi \rVert $ and taking the supremum, we get
              $$ \lVert TS \rVert = \sup_{\xi \neq 0} \frac{\lVert T(S(\xi)) \rVert}{\lVert \xi \rVert } \leq \lVert T \rVert  \lVert S \rVert . $$

              If we choose $S = T$, we arrive at $\norm{T^2} \leq \norm{T}^2$. Replacing $T$ by $T^{n-1}$, we get $\norm{T^{n}} \leq  \norm{T} ^{n}$,
              for all $n$.

    \end{enumerate}

\end{exercise}

\begin{exercise}{1.1.18}
    Let $(e_n)_{n=1}^{\infty}$ be the usual basis of $l^2(\Natural)$ and $(\alpha_n)_{n=1}^{\infty}$ a sequence in $\field$.
    Show that the operator $T : l^2(\Natural) \hookleftarrow$ with $T e_n = \alpha_n e_n$ is bounded if, and only if, $(\alpha_n)_{n=1}^{\infty}$
    is a bounded sequence. Verify that, in this case, $ \norm{T} = \sup_n |\alpha_n|$.

    Suppose $T$ is bounded. Then
    \begin{align*}
        \norm{T e_n}         & \leq \norm{T} \norm{e_n} \\
        \norm{\alpha_n e_n}  & \leq \norm{T} \norm{e_n} \\
        |\alpha_n|\norm{e_n} & \leq \norm{T} \norm{e_n} \\
        |\alpha_n|           & \leq \norm{T}
    \end{align*}
    so $\alpha_n$ is bounded.

    Reciprocally, suppose the sequence $\alpha_n$ is bounded. Consider $\xi \in l^2(\Natural)$,
    $$\norm{T\xi} = \norm{T(\sum_i c_i e_i)} = \norm{\sum_i c_i \alpha_i e_i} \leq \sup_n |\alpha_n| \norm{\sum_i c_i e_i} = \sup_n |\alpha_n| \norm{\xi}$$
    so $T$ is bounded and $\norm{T} = \sup_n |\alpha_n|$.
\end{exercise}

\begin{exercise}{1.1.19} Let $C^1 (0, 1)$ be the set of continuously differentiable real functions on $(0, 1)$, as a subspace of $L^2 (0, 1)$
    (i.e., use the norm of $L^2$). Apply the differential operator $(D\psi)(t) = \psi'(t), D : C^1 (0, 1) \to L^2 (0, 1)$, to functions $\psi_n (t) = \sin(n \pi t)$
    and conclude that D is not bounded.

    For each $n$, $\norm{\psi_n} \leq 1$. Indeed:

    $$\int_0^1 |\sin n\pi t|^2 dt = \int_0^1 \frac{|1 - \cos 2n \pi x|}{2} dt = \left[\frac{t}{2} - \frac{\sin 2 n \pi x}{4 n \pi}\right]_0^1 = 1/2. $$


    Let's consider $D \psi_n = D \sin n \pi t = n \pi \cos n \pi t$. Taking the norm,
    $$ \norm{D\psi_n} = \int_0^1 |n \pi \cos n \pi t|^2 dt = |n\pi|^2 \int_0^1 |\cos n \pi t|^2 dt$$
    $$ = |n\pi|^2 \int_0^1 \frac{1 + |\cos 2n \pi t|}{2} dt = \frac{|n\pi|^2}{2} \left[ t + \frac{\sin 2n \pi t}{2 n \pi} \right]_0^1 = \frac{|n\pi|^2}{2}. $$
    Thus $\norm{D \psi_n}  \to \infty $ as $n \to \infty$, so $D$ is not bounded.

\end{exercise}

\begin{exercise}{1.1.20}
    Show that the differential operator $D : C^{\infty} [a, b] \hookleftarrow $ is not bounded for any norm on $C^\infty[a, b]$.

    Consider the sequence of polinomials $(t^n)_{n\in\Natural}$. Then $\norm{Dt^n} = \norm{nt^{n-1}} = n \norm{t^{n-1}}$. Since we can have $n\to\infty$, $D$ can't be bounded.

\end{exercise}

\begin{exercise}{1.1.22}
    Suppose that $T_n \to T$ in $\bounded(\mathcal{N})$ and $\xi_n \to \xi$ in $\mathcal{N}$. Show that $T_n\xi_n \to T \xi$.

    Let's first derive the following inequality:
    $$ \norm{T\xi - T_n\xi_n} =  \norm{T\xi - T_n\xi + T_n\xi -T_n\xi_n }$$
    $$ \leq \norm {T\xi - T_n\xi} + \norm{T_n\xi -T_n\xi_n} $$
    Now we show that each of those terms converge to $0$:

    First, $ \norm {T\xi - T_n\xi} = \norm{(T-T_n)\xi} \leq \norm{T-T_n} \norm{\xi}$.
    Since $T_n \to T$, we can take $\norm{T-T_n} < \frac{\epsilon}{2 \norm{\xi}}$, so $\norm {T\xi - T_n\xi} < \frac{\epsilon}{2}$ (for $n = n'$).

    Similarly, $ \norm{T_n\xi -T_n\xi_n} \leq \norm{T_n} \norm{\xi - \xi_n}$, and we can take $\norm{\xi - \xi_n} < \frac{\epsilon}{2\norm{T_n}}$
    leaving $\norm{T_n\xi -T_n\xi_n} < \frac{\epsilon}{2}$ (for $n = n''$).

    We conclude, then, $\norm{T\xi - T_n\xi_n} < \epsilon$ (for $n = \max\{n', n''\}$) , which means that $T_n\xi_n \to T \xi$.


\end{exercise}

\begin{exercise}{1.1.23} Let $T \in \bounded(\mathcal{B})$. Show that, for all $t \in \field$, the operator $e^{tT}$ defined by the series
    $$ e^{tT} := \sum_{j=0}^{\infty} \frac{(tT)^j}{j!} $$
    belongs to $\bounded(\mathcal{B})$ and $\norm{e^{tT}} \leq e^{|t|\norm{T}}$.

    $$ \norm{e^{tT}} = \norm{\sum_{j=0}^{\infty} \frac{(tT)^j}{j!}} \leq \sum_{j=0}^{\infty} \frac{\norm{(tT)^j}}{j!}
        \stackrel{*}{\leq} \sum_{j=0}^{\infty} \frac{\norm{tT}^j}{j!} = \sum_{j=0}^{\infty} \frac{(|t|\norm{T})^j}{j!} = e^{|t|\norm{T}}.$$

    * by 1.1.13.(b).


\end{exercise}

\begin{exercise}{1.1.24}
    Let $T \in \bounded (\mathcal{B})$, with $\norm{T}<1$. Show that the operator defined by the series $S = \sum_{j=0}^\infty T^j$ belongs to $\bounded (\mathcal{B})$
    and $S = (1-T)^{-1}.$

    $$\norm{S} = \norm{\sum_{j=0}^\infty T^j} \leq \sum_{j=0}^\infty \norm{T^j} \leq \sum_{j=0}^\infty \norm{T}^j.$$
    Since $\norm{T} < 1$, the right side is a geometric series and converges.

    To verify that $S = (1-T)^{-1}$,
    $$ S (1-T) = \sum_{j=0}^\infty T^j (1-T) = \sum_{j=0}^\infty T^j - T^{j+1}$$
    $$ = \sum_{j=0}^\infty T^j - \sum_{j=1}^\infty T^j = T^0 = 1.$$

\end{exercise}

\begin{exercise}{1.1.30}
    Show that the dual of $l^p$ is $l^q$, with $1 < p < \infty$ and $1/p + 1/q = 1$.

    \textcolor{red}{nao manjei}

\end{exercise}

\begin{exercise}{1.1.35}
    Let $S_l : l^2 (\Natural) \hookleftarrow$ be the shift $S_l (\xi_1 , \xi_2 , \xi_3 , \dots) = (\xi_2 , \xi_3 , \xi_4 , \dots)$
    and $T_n = S_l^n$. Find $\norm{T_n\xi}$, and the limit operator described in the Banach-Steinhaus theorem.

    Given $T_n$, let $e_{n+1}$ be the nth element of the canonical basis of $l^2$. Also, $\norm{e_{n+1}} = 1$ and $\norm{T_n e_{n+1}} = 1$, so $\norm{T_n} \geq 1$.

    Now, given an arbitrary $\xi = (\xi_1, \xi_2, \dots) \in l^2$ with $\norm{\xi}=1$. This means that $\norm{\xi} = \sum_{i=1}^\infty \xi_i^2 = 1$.
    Thus, $\norm{T_n \xi} = \sum_{i=n+1}^\infty \xi_i^2 \leq 1$ necessarily. So $\norm{T_n} \leq 1$. In conclusion, $\norm{T_n}=1$.

    By the Banach-Steinhaus theorem there exists the limit $T\xi = \lim_{n\to \infty} T_n \xi$.
    $$\norm{T\xi} = \norm{\lim_{n\to \infty} T_n \xi} = \lim_{n\to \infty} \norm{T_n \xi} = \lim_{n\to \infty} \sum_{i=n+1}^\infty \xi_i^2 = \lim_{n\to \infty} \xi_i = 0$$
    Therefore $T\xi = 0$.

\end{exercise}

\begin{exercise}{1.1.38}

    Verify that $C_p [0, 2\pi]$ is a Banach space, and also the validity of the expression for a partial sum for the Fourier series used in the proof of Corollary 1.1.37.


\end{exercise}

\begin{exercise}{1.1.39} (Cauchy Schwarz inequality)

\end{exercise}

\begin{exercise}{1.1.42}
    If $f \in H^*$, what is the dimension of $\kernel(f)^\perp$?

    \textcolor{red}{1 I think?}

    $\dim (\kernel(f)^\perp )= \dim \field = 1.$

\end{exercise}

\subsection{Closed Operators}


\begin{exercise}{1.2.3}
    Show that $T : l^1 (\Natural) \hookleftarrow$ given by $$T (\xi_1, \xi_2, \xi_3, \dots ) = (\xi_1 /1, \xi_2 /2,\xi_3 /3, \dots )$$
    is linear, continuous and invertible, but its inverse $T^{-1}$, defined on the
    range of $T$, is not a continuous operator.


    Suppose and arbitrary $\xi \in l^1(\Natural)$.
    $$\norm{\xi} = \sum_{n=1} \xi_n = \xi_1 + \sum_{n=2} \xi_n.$$
    $$ \norm{T\xi} = \sum_{n=1} \xi_n/n = \xi_1 + \sum_{n=2} \xi_n/n $$
    So,
    $$ \norm{T\xi} \leq \xi_1 + \frac{1}{2} \sum_{n=2} \xi_n \leq \norm{\xi}$$
    and we can conclude that $T$ is bounded.

    Let's consider now its inverse $ T(1\xi_1, 2\xi_2, 3\xi_3, \dots) = (\xi_1, \xi_2,\xi_3, \dots)$. Let $e_n$ be the sequence in $l^1$ with $1$ in its n-th coordinate and
    $0$ everywhere else. Clearly $\norm{e_n} = 1 \forall n$, but $\norm{Te_n} = n$, which goes to infinity as $n\to\infty$. Thus, $\sup T(e_n) = \infty$, and $T$ in not bounded.

\end{exercise}

\begin{exercise}{1.2.9}


    \begin{enumerate}

        \item Show that $\banach_1 \times \banach_2$ with the norm
              $$\norm{(\xi, \eta)} = \left(\norm{\xi}^2_{\banach_1} + \norm{\eta}^2_{\banach_2}\right)^{\frac{1}{2}}$$
              defined above is a Banach space.

              Consider a cauchy sequence $(\xi,\eta)_n=(\xi_n, \eta_n)$. This means that, given $\epsilon>0$, there exists $n_0$ such that, if $n, m > n_0$,
              $$ \norm{(\xi,\eta)_n - (\xi,\eta)_m} = \left(\norm{\xi_n - \xi_m}^2_{\banach_1} + \norm{\eta_n-\eta_m}^2_{\banach_2}\right)^{\frac{1}{2}} < \epsilon$$
              $$ \left(\norm{\xi_n-\xi_m}_{\banach_1} + \norm{\eta_n - \eta_m}_{\banach_2}\right)^2 \leq \norm{\xi_n-\xi_m}^2_{\banach_1} + \norm{\eta_n - \eta_m}^2_{\banach_2} < \epsilon^2$$
              $$ \norm{\xi_n-\xi_m}_{\banach_1} + \norm{\eta_n - \eta_m}_{\banach_2} < \epsilon$$
              Since the norms are necessarily positive, and are both smaller than arbitrary $\epsilon$, we can conclude that $\xi_n$ and $\eta_n$ are cauchy
              in $\banach_1$ and $\banach_2$ respectively, and therefore $\xi_n \to \xi$ and $\eta_n \to \eta$.

              Thus, our candidate for the convergence of $(\xi_n,\eta_n)$ is $(\xi,\eta)$. Let's investigate this convergence.
              $$ \norm{(\xi_n,\eta_n) - (\xi,\eta)} = \norm{(\xi_n-\xi,\eta_n-\eta)} $$
              $$ = \left(\norm{\xi_n-\xi}^2_{\banach_1} + \norm{\eta_n-\eta}^2_{\banach_2}\right)^{\frac{1}{2}}$$
              From the individual convergences of $\xi$ and $\eta$,
              $$ \norm{(\xi_n,\eta_n) - (\xi,\eta)} < \sqrt{\epsilon^2 + \epsilon^2} = \sqrt{2} \epsilon. $$
              We've shown then that any cauchy sequence converges, and so this space is Banach.


        \item Show that $T$ is a closed operator iff $\dom T$ with the graph norm is a Banach space.

              \textcolor{red}{(Should we consider here $T: \banach_1 \to \banach_2$ ? If so:)}
              $T$ is closed if its graph is a closed subset of $\banach_1 \times \banach_2$. A closed subset of a complete space is complete.
              A complete subset of a complete space is closed.
    \end{enumerate}

\end{exercise}

\begin{exercise}{1.2.10}
    Verify that $\graph(T)$ is a vector subspace of $\normed_1 \times \normed_2$ and the equivalence quoted in the above definition of closed operator.

    Let $\xi, \psi \in \dom T$ and $\alpha \in \field$. Of course, $(\xi, T\xi), (\psi, T\psi) \in \graph(T)$.
    Since $\xi+\alpha \psi \in \dom T$ and $(\xi+\alpha \psi, T(\xi+\alpha \psi)) \in \graph(T)$.
    $$ (\xi+\alpha \psi, T(\xi+\alpha \psi)) = (\xi+\alpha \psi, T\xi+ \alpha T \psi) = (\xi, T\xi) + \alpha(\psi, T\psi) \in \graph(T)$$
    So $\graph(T)$ is a vector subspace.


    Now to show the equivalence, suppose $(\xi_n, T\xi_n)\in \normed_1 \times \normed_2$ converges to $(\xi, \eta)$. It's clear that it can only happen if and only if
    $\xi_n \to \xi$ and $T\xi_n \to \eta$ in $\normed_1, \normed_2$ respectively.
    Then, the graph $\graph(T)$ is closed if and only if it contains the limit  $(\xi_n, T\xi_n) = (\xi,\eta)$ (since our sequence is arbitrary).
    By the uniqueness of limits, $T(\xi) = \eta.$



\end{exercise}

\begin{exercise}{1.2.12}
    Consider the linear operator $T : \dom T \subset \normed_1 \to \normed_2$, and let $\pi_1 : \graph(T) \to \dom T$
    and $\pi_2 : \graph(T) \to \rng T$ be the natural projections $\pi_1 (\xi, T \xi) = \xi$
    and $\pi_2 (\xi, T \xi) = T\xi$, for $\xi \in \dom T$. Show that such projections are continuous linear operators.

    It's easy to see that, given $(\xi, t\xi) \in \graph T$,
    $$\norm{\pi_1 (\xi, t\xi)}_{\normed_1} = \norm{\xi}_{\normed_1}\leq \norm{(\xi, t\xi)}_{\graph}=\sqrt{\norm{\xi}_{\normed_1}^2 + \norm{T\xi}_{\normed_2}^2}$$
    and
    $$\norm{\pi_2 (\xi, t\xi)}_{\normed_2} = \norm{T\xi}_{\normed_2}\leq \norm{(\xi, t\xi)}_{\graph}=\sqrt{\norm{\xi}_{\normed_1}^2 + \norm{T\xi}_{\normed_2}^2}$$
    or more simply, because the projections are (weak) contractions.



\end{exercise}

\begin{exercise}{1.2.14}
    If $\dim \normed_1 < \infty$, show that every linear operator $T : \dom T \subset \normed_1 \to \normed_2$ is closed.

    Since $\dim \normed_1 < \infty$, $\normed_1$ is complete. The range of $\normed_1$ is also a finite normed vector space, and therefore complete. By the proposition 1.2.13
    (Any operator $T \in \bounded(\banach_1,\banach_2)$ is closed.), $T$ is closed.


\end{exercise}

\begin{exercise}{1.2.16}
    If $\normed \subset \banach$, show that $T \in \bounded(\normed , \banach)$ is closed if, and only if, $\normed$ is a Banach space.

    If $\normed$ is a Banach space, then $T$ is closed by the proposition 1.2.13.
    On the other hand, if $\normed$ is not complete, there is a cauchy sequence $(\xi_n)$ that converges to $\xi$ in $\banach \setminus \normed$.
    The sequence $T\xi_n$ is also cauchy (because $T$ is uniformly continuous) and converges in $\banach$. But since $\xi \notin \normed$, $T$ is not closed.


\end{exercise}

\begin{exercise}{1.2.19}
    From Example 1.2.18, show that if $(\psi_j )_{j=1}^\infty \subset C^1[0, \pi]$ is such that
    the series $\psi(t) = \sum_{j=1}^\infty \psi_j (t)$ and $\varphi(t) = \sum_{j=1}^\infty \psi_j' (t)$ converge uniformly, then $\psi$
    is continuously differentiable and $\varphi=\psi'$.

    \textcolor{red}{Fazer depois.}


\end{exercise}

\begin{exercise}{1.2.24}
    Let $E$ be a subspace of $\normed_1 \times \normed_2$. Show that $E$ is the graph of a linear operator if, and only if, $E$ does not contain any element of the form $(0, \eta)$,
    with $\eta \neq 0$.

    Any linear operator takes $0$ to $0$, therefore if $E$ is a graph of a linear operator, it cannot possibly contain $(0, \eta \neq 0)$.
    Reciprocally, suppose it doesn't contain such element. $E$ is a linear subspace, and so are its projections onto $\normed_1$ and $\normed_2$: $\pi_{1}(E)$ and $\pi_{2}(E)$.
    Then we can define a linear transformation by relating the elements of the basis of $\pi_{1}(E)$ and $\pi_{2}(E)$ (as long as $\dim \pi_1 > \dim \pi_2$, right?
    \textcolor{red}{CHECK THIS}).

\end{exercise}

\begin{exercise}{1.2.26}
    Show that $X$ is a core of the closed operator $T$ iff $\{(\xi, T\xi): $ $\xi \in X\}$ is dense in $\graph(T)$.

    $\{(\xi, T\xi): \xi \in X\}$ is the graph of $T|_X$. By definition, $\graph(T|_X)$ is dense in $\graph(T)$ iff $\closure{\graph(T|_X)} = \graph(T)$,
    which is the definition of $X$ as a core of $T$.

\end{exercise}

\begin{exercise}{1.2.28}
    Show that $T$ is a closed operator acting in $\hilbert$ iff $\dom T$ with the graph inner product of $T$, given by
    $\inner{\eta}{\xi}_T := \inner{T\eta}{T\xi} + \inner{\eta}{\xi}$, is a Hilbert space.
    This inner product generates a graph norm (Definition 1.2.7) and the corresponding orthogonality will be denoted by $\perp_T$.

    The Hilbert space is a Banach space with the norm generated by its inner product $\langle , \rangle_{\hilbert}$.
    It suffices to show that the inner product $\langle , \rangle_T$ defined above generates the graph norm, and the rest follows from exercise 1.2.9.(b).
    The norm induced from the inner product is
    $$\norm{\xi}_T = \sqrt{\inner{\xi}{\xi}}_T = \sqrt{\inner{T\xi}{T\xi}_{\hilbert} + \inner{\xi}{\xi}_{\hilbert}}= \left(\norm{T\xi}^2 + \norm{\xi}^2\right)^{\frac{1}{2}}$$
    which is precisely the graph norm.

\end{exercise}

\subsection{Compact Operators}

\begin{exercise}{1.3.1}
    Show that if $A \subset (X, d)$ is precompact, then $A$ is totally bounded and, so, bounded.

    If $A$ is precompact, then $\closure{A}$ is compact. We can think of a open covering of $\closure{A}$ such as $\cup_{\xi \in A} \openball{\xi}{\varepsilon}$,
    and since $\closure{A}$ is compact it must have a finite subcovering (of open balls in $A$), which contains $A$ by extension. Consequently, $A$ is totally bounded and bounded.

\end{exercise}

\begin{exercise}{1.3.2}
    If $A \subset (X, d)$ is totally bounded, show that, for all $\varepsilon > 0$, $A$ is in the union of a finite number of open balls of radii $\varepsilon$ centered at points of $A$.
    Conclude then that a totally bounded set is separable with the induced topology, that is, it contains a countable dense subset.

    $A$ is totally bounded, so there is a finite number of open balls of radii $\varepsilon/2 > 0$ whose union contain $A$.
    However, some of those balls could be centered at points outside $A$.
    Consider one of them, centered at $b \in X\setminus A$: They must contain a point $a \in A$ (otherwise they could be discarded);
    this point $a$ is at a distance $d(a,b)<\varepsilon/2$;
    an open ball centered at $a$ with radius $\varepsilon$ is sure to contain $\openball{b}{\varepsilon/2}$ (this is easily seen from the triangle inequality of $d$).
    Therefore, if we replace every ball with radius $\varepsilon/2$ by a ball with radius $\varepsilon$, changing the center to a point in $A$ if necessary,
    we construct a set of balls, with centers in $A$, that contain $A$, such as we wanted.

    For the second part, we can conceive a finite collection $E_\varepsilon$ of open balls with radius $\varepsilon > 0$ and center in $A$,
    and let $C_\varepsilon$ be the collection of these centers. Then,
    $$ C_\varepsilon \cup C_{\varepsilon/2} \cup C_{\varepsilon/4} \cup C_{\varepsilon/8} \dots$$
    is a countable dense subset in $A$ (because any point in $A$ is contained in an open ball around one of these points, for any $\varepsilon$) and so $A$ is separable.


\end{exercise}

\begin{exercise}{1.3.6}
    If $\dim \normed = \infty$, show that the identity operator $\id : \normed \hookleftarrow$ is not compact (use, for instance, Riesz’s Lemma 1.6.2).

    To show that $\id$ isn't compact, we must provide a set in $\normed$ that is bounded but isn't precompact. Well, if the dimension is infinite, we can choose
    the unitary open ball $\openball{0}{1}$ as our bounded set (it is clearly bounded). However, it is not precompact because notably the closed ball
    $ \closure{\openball{0}{1}} = \closedball{0}{1}$ is not compact on infinite dimension.

\end{exercise}

\begin{exercise}{1.3.12}
    Let $T : \normed_1 \to \normed_2$ linear. Show that it is compact if, and only if, $T \ball{0}{1}$ is precompact in $\normed_2$.

    Clearly $\ball{0}{1}$ is bounded, and so $T \ball{0}{1}$ is precompact. Conversely, if $\ball{0}{1}$ is precompact, $\closedball{0}{1}$ is compact, and so $\normed_2$
    must be finite dimensional (otherwise, the closed ball wouldn't be compact). Since $\normed_2$ is finite dimensional, $T$ is finite rank, thus compact.

    \textcolor{red}{Essa recíproca está certa? Além disso, então todo operador compacto tem rank finito? Essa dúvida tem a ver com o corolário 1.3.14.}


\end{exercise}

\begin{exercise}{1.3.16}
    Show that in $\bounded(\normed_1 , \normed_2 )$, the three kinds of limits defined above
    are well defined and unique (if they exist, of course). Moreover, verify that the
    uniform convergence $\Rightarrow$ strong convergence $\Rightarrow$ weak convergence, and with the
    same limits.

    Suppose $T_n \to T$, uniformly. Then given $\xi$ there exists $n$ such that $\norm{T_n - T} < \frac{\varepsilon}{\norm{\xi}}$.
    Therefore, $\norm{T_n \xi - T \xi} \leq \norm{T_n - T}\norm{\xi} < \varepsilon$, so $T_n \strongly T$.

    Suppose now only the strong convergence $T_n \strongly T$. Let $f \in \normed_2^*$ and $\xi \in \normed_1$. There exists $n$ such that $T_n \xi - T \xi < \frac{\varepsilon}{\norm{f}}$.
    The functional $f$ is of course bounded, thus $|f(T_n \xi) - f(T \xi)| \leq \norm{f} \norm{T_n \xi - T \xi} < \varepsilon$, and $T_n \weakly T$.

\end{exercise}

\begin{exercise}{1.3.18}
    Show that the sequence of operators $T_n : l^2 (\Natural) \hookleftarrow$
    $$ T_n \xi = \left(0, 0, \dots, 0, \xi_{n+1} , \xi_{n+2} , \xi_{n+3}, \dots \right)$$
    converges strongly to zero, but does not converge uniformly.


    The operator $T_n$ erases the $n$ first entrees of the vector it's applied to, so it's easy to see that it converges strongly to 0: Given $\xi \in l^2$, $\norm{\xi} = \sum_{k=1}^\infty \xi_k$. Then, $\norm{T_n \xi} = \sum_{k=n+1}^\infty \xi_k$ and thus,
    $$\lim_{n\to\infty} \sum_{k=n+1}^\infty \xi_k^2 = \lim_{n\to\infty} \left(\sum_{k=1}^\infty \xi_k^2 - \sum_{k=1}^{n} \xi_k^2 \right)$$
    $$ = \sum_{k=1}^\infty \xi_k^2 - \lim_{n\to\infty}\sum_{k=1}^{n} \xi_k^2= \sum_{k=1}^\infty \xi_k^2 - \sum_{k=1}^\infty \xi_k^2 = 0.$$

    So $T_n \strongly T$. However, for every $T_n$, there is the canonical basis vector $e_{n+1}$ for which $\norm{T_n e_{n+1}} = 1$. On the other hand, the 0-vector   has norm $0$, of course.
    So $T_n$ can't converge in norm to $0$.

\end{exercise}

\begin{exercise}{1.3.19}
    Show that the sequence of operators $T_n : l^2 (\Natural) \hookleftarrow$
    $$ T_n \xi = \left(0, 0, \dots, 0, \xi_{1} , \xi_{2} , \xi_{3}, \dots \right)$$
    (n-shift) converges weakly to zero, but does not converge strongly.

    The tricky part here is that the shift operator preserves the entries of the vectors it's applied to, it just shifts them, so it preserves the norm.
    So if given a $\xi \in l^2$,
    $$ \norm{T_n \xi} = \sum_{k=1}^n 0^2 +  \sum_{k=n+1} \xi_{k-n}^2 = \sum_{k=1}\xi_k^2 = \norm{\xi}.$$
    For this reason, it doesn't converge strongly.

    However, we can show that it converges weakly. We just have to notice this aspect of the linear functionals. Given $f \in (l^2)^*$, $\xi \in l^2$, and a basis $(e_n)$ of $l^2$,
    $$ |f(\xi)| = |f(\sum \xi_n e_n)| = |\sum \xi_n f(e_n)| < \infty.$$
    From this, it follows that $f(e_n) \to 0$. (\textcolor{red}{To show this, suppose it didn't. ...})

    Then,
    $$ \lim_n |f(T_n \xi)| = \lim_n |\sum_k \xi_k f(T_n e_k)| = |\sum_k \xi_k \lim_n f(e_{k+n})| = 0$$


\end{exercise}

\begin{exercise}{1.3.21}
    Show that every orthonormal sequence in a Hilbert space converges weakly to zero and has no strongly convergent subsequence.

    \textcolor{red}{pedir arrego na primeira parte}

    Let $(T_n)$ be a orthonormal sequence in $\hilbert$. The distance between any two terms is
    $$\norm{T_n - T_m} = \inner{T_n - T_m}{T_n - T_m} = \inner{T_n}{T_n} - \inner{T_n}{T_m} - \inner{T_m}{T_n} + \inner{T_m}{T_m}$$
    $$ \norm{T_n} + 0 + 0 + \norm{T_m} = 2$$
    So it can't possibly converge strongly.

    \textcolor{red}{estou tendo dificuldade com convergencia fraca. Por exemplo, a proposicao 1.3.22: nao entendi. Uma sequencia fracamente convergente e limitada?}




\end{exercise}

\begin{exercise}{1.3.26}
    Let $T \in \bounded(\hilbert)$, with $\hilbert$ separable. Show that there is a sequence $(T_n)$ of finite rank operators which converges strongly to $T$,
    that is, $T_n \strongly T$.

    The space $\hilbert$ is separable, therefore it has a enumerable basis $(e_n)$. By the parseval identity, for any $\xi \in \hilbert$,
    $$ \norm{\xi} = \sum_{k=1}^\infty |\inner{e_k}{\xi}|^2$$
    and since this series has got to converge, $\lim_{k\to\infty} |\inner{e_k}{\xi}| = 0$.

    Now consider the finite rank operator $T_n = TP_n$, where $P_n$ is the projection onto the subspace given by $\linspan(e_1,\dots,e_n)$.


    $$ \norm{T_n \xi - T\xi}^2 = \sum_{k=1}^\infty |\inner{e_k}{(T_n - T)\xi}|^2$$
    $$ = \sum_{k=1}^\infty |\inner{e_k}{T_n\xi} - \inner{e_k}{T\xi}|^2 =
        \begin{cases}
            0                     & k \leq n \\
            |\inner{e_k}{T\xi}|^2 & k > n
        \end{cases}
    $$

    $$ = \sum_{k=n}^\infty |\inner{e_k}{T\xi}|^2$$
    so,
    $$ \lim_{n \to \infty} \sum_{k=n}^\infty |\inner{e_k}{T\xi}|^2 = 0$$
    and $T_n \strongly T$.


\end{exercise}

\begin{exercise}{1.3.31}
    Show that a precompact set (compact) in $(C[a, b], \norm{.}_\infty )$ is precompact (compact) in $L^2[a, b]$. This occurs because the identity map 
    $\id : (C[a, b]\norm{.}_\infty) \to L^2 [a, b]$ is continuous.

    The bounded operator $\id$ is continuous, therefore the image of a compact set is compact. Also, $f\left(\closure{A}\right) = \closure{f(A)}$, because of sequential continuity.



\end{exercise}

\subsection{Hilbert-Schmidt Operators}

\subsection{The spectrum}

\subsection{Spectra of Compact Operators}

\section{Adjoint Operator}

\end{document}
