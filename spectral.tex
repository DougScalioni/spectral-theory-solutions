\documentclass{article}
\usepackage[utf8]{inputenc}
\usepackage{amssymb}
\usepackage{amsmath}
\usepackage{amsthm}



\theoremstyle{remark}
\newtheorem{innerex}{Exercise}
\newenvironment{exercise}[1]
  {\renewcommand\theinnerex{#1}\innerex}
  {\endinnerex}

\DeclareMathOperator{\domain}{dom } \newcommand{\dom}{\domain}

\DeclareMathOperator{\range}{rng } \newcommand{\rng}{\range}

\DeclareMathOperator{\nullspace}{N} \newcommand{\kernel}{\nullspace}

\renewcommand{\labelenumi}{\alph{enumi})}


\begin{document}


\title{Intermediate Spectral Theory and Quantum Dynamics \\ César R. de Oliveira \\ Notes and Solutions}
\author{Douglas Scalioni Domingues} 
\date{April 2021}

\maketitle

\section{Linear Operators and Spectra}

\subsection{Bounded Operators}

\begin{exercise}{1.1.5}
    Let $T : \dom{T} \subset X \rightarrow Y$ be a linear operator. Verify the following
    items:

    \begin{enumerate}
        \item The range of T , $\rng T := T (\dom T) \subset Y$ , and the kernel (or null space) of T,
              $\kernel(T) := \{\xi \in \dom T : T \xi = 0\}$, are vector spaces.
              

              \underline{rng}: Let $X$, $Y$ be in the range of $T$ and $\alpha \in \mathbb{F}$. There
              exist $x$ and $y$ in $\dom T$ such that $Tx=X$ and $Ty = Y$. Since $\dom T$ is a vector
              subspace, $\alpha x + y$ is also in $\dom T$, and therefore $T(\alpha x + y)$ exists. $T$
              is linear, then $T(\alpha x + y) = \alpha T(x) + T(y)$.
              

              \underline{kernel}:
              Let $x, y \in \textrm{ker}(T)$ and $\alpha \in \mathbb{F}$. Then $T(\alpha x+y) = \alpha T(x) + T(y) = 0 + 0 = 0$,
              and $\alpha x + y \in \textrm{ker}(T)$.
              

        \item If the dimension $\dim(\dom T) = n < \infty$, then $\dim(\rng T ) \leq n$.
              

              The dimension of a vector space is given by the number of elements in its basis. Suppose $B = \{ b_1, ..., b_n\}$
              is a basis of dom $T$ with $n$ elements. Given an arbitrary element x of $\dom T$, we have $x = a_1 b_1 + ... + a_n b_n$.
              The range of $T$ is completely determined by its action on the basis of $\dom T$:
              \begin{math}
                  T(x) = T(a_1 b_1 + ... + a_n b_n) = a_1 T(b_1) + ... + a_n T(b_n)
              \end{math}
              Clearly, the set $\{T(b_1) + ... + T(b_n)\}$ spans rng $T$, and therefore  $\dim(\rng T) \leq n$.
              

        \item The inverse operator of $T$, $T^{-1}: \rng T$ $\rightarrow \dom T$, exists if, and only if, $T\xi = 0 \Rightarrow \xi = 0$
              and, in case it exists, it is also a linear operator.
              

              For $T$ to have an inverse it needs to be a bijection, just like any other function. In particular, $T$ has to be injective.
              $T(a) = T(b)$ implies that $a=b$. But $T(a) = T(b) \Rightarrow T(a) - T(b) = T(a-b) = 0$

    \end{enumerate}

\end{exercise}



\end{document}
